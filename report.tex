\documentclass[12pt,letterpaper,english,bibliography=totocnumbered]{scrartcl}

\usepackage{indentfirst}
\usepackage{appendix}
%\usepackage{fullpage}
%\usepackage{subfiles}
\usepackage[T1]{fontenc}
\usepackage[latin9]{inputenc}
\usepackage{color}
\usepackage{babel}
\usepackage{verbatim}
\usepackage[unicode=true,pdfusetitle,
 bookmarks=true,bookmarksnumbered=false,bookmarksopen=false,
 breaklinks=true,pdfborder={0 0 0},pdfborderstyle={},backref=false,colorlinks=true]
 {hyperref}
\hypersetup{linkcolor=blue,citecolor=blue,urlcolor=blue}

\usepackage{booktabs}
\usepackage{multirow}
\usepackage{adjustbox}
\usepackage{threeparttable}
\usepackage[table]{xcolor}

\usepackage[backend=biber,maxbibnames=99]{biblatex}
\usepackage{csquotes}
\addbibresource{references.bib}

% Prevent page breaks within paragraphs
% https://tex.stackexchange.com/questions/21983/how-to-avoid-page-breaks-inside-paragraphs
\widowpenalties 1 10000

\begin{document}

\titlehead{USDA APHIS Grant AP17PPQFO000C312\\
Final Report\\
Report ID: AP17PPQFO000C312-PE-Final-19\\
Performance Period: August 1, 2017 - July 31, 2019}
\title{Coconut Rhinoceros Beetle Biological Control}
\author{Aubrey Moore, University of Guam}
\maketitle

% The following line provides a link to the source code for this document

\begin{center}
\url{https://github.com/aubreymoore/FB17-Final-Report/raw/master/report.pdf}
\end{center}

\newpage
\section*{In a Nut Shell}

\fcolorbox{black}{lightgray}{
	\begin{minipage}{\textwidth}
		\begin{itemize}
			
			\item The primary objective of this project was to find an isolate of \textit{Oryctes rhinoceros} nudivirus (OrNV) which can be used as an effective biological control agent for CRB-G biotype of coconut rhinoceros beetle (CRB). Laboratory bioassays of four isolates indicated that OrNV isolate V23B is pathogenic for CRB-G and a potential biocontrol agent for this pest.
			
			\item A secondary objective of this project was to develop a CRB damage monitoring system. A digital image analysis system was developed to detect and quantify V-shaped cuts to fronds and coconut palm mortality caused by CRB. The heart of this system is an object detector, trained by deep learning technology, which locates CRB damage symptoms on frames from georeferenced roadside video surveys. This object detector will be used to automate detection, quantification and to map changes in CRB damage over time and space.
			
			\item Uncontrolled outbreaks of CRB-G is a major problem for Pacific islands. This pest is causing major damage to the oil palm plantations in Papua New Guinea and to coconut plantations in the Solomon Islands....
			
	\end{itemize}
	\end{minipage}
}

\newpage
\tableofcontents{}

\newpage

\section{Background}

The major goal of this project is to find an effective biological
control agent for coconut rhinoceros beetle biotype G (CRB-G).

Prior to arrival of CRB-G on Guam during 2007, coconut rhinoceros beetle
infestations of Pacific islands were readily controlled by classical
biological control using \textit{Oryctes} nudivirus (OrNV). Following a lack
of response to release of OrNV on Guam, research showed that the Guam
CRB population is a genetically distinct virus-resistant biotype which
has become known as CRB-G \cite{marshall_new_2017-1}. This biotype is highly invasive and is
causing massive damage to coconut and oil palms in Papua New Guinea
and the Solomon Islands. CRB-G has also invaded Oahu and Rota. Eradication
attempts have been launched on these two islands.

Additional goals for this project are to establish a CRB damage survey to evaluate efficacy of biocontrol and other tactics, and to maintain and facilitate collaboration with other Pacific island entomologists working to find solutions for CRB-G management.

\section{Staffing}

Staff for this project currently comprises of only 2 people: the PI, Dr. Aubrey Moore, and a post-doc, Dr. James Grasela.
\begin{itemize}

    \item Dr. James Grasela, an insect pathologist, has been hired for a term of 2 years with a grant from Department of Interior, Office of Island Affairs.

    \item Ian Iriarte, a graduate student working on this project, resigned to accept a permanent job. Search for a replacement is under way.

\end{itemize}


\section{Bioassays to Detect Candidate Biocontrol Agents for CRB-G}

\subsection{Bioassay Results}

Recent bioassay results are attached in Appendix \ref{recent-bioassay-results}.
During laboratory bioassays, we dosed CRB-G with samples of OrNV isloates, and observe mortality and changes in mass for one month. Each beetle is kept in isolation and individual records are stored in a laboratory information system (Section \ref{lims}).

% Please add the following required packages to your document preamble:
% \usepackage{booktabs}
% \usepackage{multirow}
\begin{table}[h]
		\begin{adjustbox}{width=\columnwidth,center}
			
	 	\begin{threeparttable} 
	\caption{\textit{Oryctes rhinoceros} nudivirus (OrNV) bioassay results summary.}
	 		
			
	\begin{tabular}{ l l l c c c c }
		\toprule
		OrNV                  & bioassay                                        & method\tnote{1} & beetles & replicates & virus                           & inactivated                     \\
		isolate               &                                                 &                 &         &            & mortality (\textit{p})\tnote{2} & virus                           \\
		                      &                                                 &                 &         &            &                                 & mortality (\textit{p})\tnote{3} \\ \bottomrule
		DUG42                 & DUG42\cite{moore_bioassay_2019}                 & injection       & 30      & 2          & 40\% (0.65)                     & 40\% (0.65)                     \\ \midrule
		\multirow{2}{*}{MALB} & MALB\cite{moore_bioassay_2019-6}                & injection       & 30      & 2          & 50\% (0.37)                     & \hphantom{0}0\% (1.00)          \\
		                      & MALBperOS\cite{moore_bioassay_2019-7}           & per os          & 13      & 1          & -60\% (1.00)                    & 20\% (1.00)                     \\ \midrule
		\multirow{2}{*}{PNG}  & PNG\cite{moore_bioassay_2019-2}                 & injection       & 81      & 4          & \cellcolor{yellow}{90\% (0.00)} & \hphantom{0}5\% (1.00)          \\
		                      & PNGperOS\cite{moore_bioassay_2019-9}            & per os          & 21      & 1          & \hphantom{0}0\% (1.00)          & \hphantom{0} 0\% (1.00)         \\ \midrule
		\multirow{4}{*}{V23B} & V23B\cite{moore_bioassay_2019-3}                & injection       & 66      & 4          & \cellcolor{yellow}{88\% (0.00)} & \hphantom{0}0\% (1.00)          \\
		                      & V23BperOS\cite{moore_bioassay_2019-5}           & per os          & 32      & 2          & 80\% (0.07)                     & 20\% (0.69)                     \\
		                      & V23-large\_bioassay\cite{moore_bioassay_2019-4} & per os          & 53      & 1          & \cellcolor{yellow}{42\% (0.00)} & -                               \\
		                      & V23\_perOSIN\cite{moore_bioassay_2019-1}        & per os          & 16      & 1          & 60\% (0.06)                     & -                               \\ \bottomrule
	\end{tabular}
	\begin{tablenotes}[para]
		\item[1] Adult beetles were dosed either by direct injection of virus suspension into the haemocoel or by applying a droplet containing virus to mouthparts. \\ 
		\item[2] Percent mortality in beetles treated with virus, adjusted for untreated control mortality; 
		number in parentheses is the \textit{p}-value resulting from a Fisher's exact test of significant difference between mortality of treated and untreated beetles. \\
		\item[3] Percent mortality in beetles treated with heat inactivated virus, adjusted for untreated control mortality; 
		number in parentheses is the \textit{p}-value resulting from a Fisher's exact test of significant difference between mortality of treated and untreated beetles. 
	\end{tablenotes}
	 		
\end{threeparttable}
	\end{adjustbox}
\end{table}

\begin{comment}

\clearpage
\begin{table}
	\begin{threeparttable}
		\caption{Sample ANOVA table}
		\begin{tabular}{lllll}
			\toprule
			Stubhead & \( df \) & \( f \) & \( \eta \) & \( p \) \\
			\midrule
			&     \multicolumn{4}{c}{Spanning text}     \\
			Row 1    & 1        & 0.67    & 0.55       & 0.41    \\
			Row 2    & 2        & 0.02    & 0.01       & 0.39    \\
			Row 3    & 3        & 0.15    & 0.33       & 0.34    \\
			Row 4    & 4        & 1.00    & 0.76       & 0.54    \\
			\bottomrule
		\end{tabular}
		\begin{tablenotes}
			\small
			\item This is where authors provide additional information about
			the data, including whatever notes are needed.
		\end{tablenotes}
	\end{threeparttable}
\end{table}

\end{comment}








\clearpage
\subsection{CRB Rearing}

Currently, experimental beetles are field-collected on Guam as prepupae, pupae or adults from breeding sites or as adults from pheromone traps.  Each beetle is given a serial number and is reared individually in a Mason jar stored in one of three environmental chambers set for 30 degrees Celsius, 80\% relative humidity and 12 hour photoperiod. Adult beetles are feed weekly with a slice of banana. 

\subsection{Laboratory Information Management System}\label{lims}

We have developed an online database which we are using as a laboratory information management system for maintaining individual records for our experiment beetles in our rearing program \cite{moore_coconut_2019-1}. This system was developed using the \href{http://www.web2py.com/}{web2py python web framework} and it is available on the web at \url{http://aubreymoore.pythonanywhere.com/rearing}. 

\subsection{Acquisition of an OrNV isolate from Taiwan}

During May 2019, Moore made a trip to Taiwan to receive 80 adult coconut rhinoceros
beetles collected by Dr. Frank Hsu of the Taiwan Normal University. The Taiwan population is of special interest because specimens in a previous collection
were all determined to be CRB-G and 82\% of these tested positive for OrNV \cite{watanabe_survey_2016-1}.

Preliminary bioassays of virus isolated from this collection did not result in significant mortality. Beetle specimens and virus samples have been sent to AgResearch New Zealand for molecular analysis.

\subsection{Acquisition of a Virus-Susceptible CRB Biotype for Comparative Bioassays}

Since discovery of the CRB-G biotype on Guam \cite{marshall_new_2017-1}, we have been operating under the hypothesis that this biotype is significantly more tolerant (resistant) to pathogenic effects of OrNV isolates previously used as effective biocontrol agents for CRB invading Pacific Islands. It has also been hypothesized that CRB-G has different behavioral characteristics, such as a significantly reduced attraction to oryctalure. However, comparative laboratory bioassays have not been performed to test these hypotheses.

We applied for and have been granted a USDA-APHIS import permit for live CRB which will allow us to establish a laboratory colony of CRB from presumed non-virus-resistant populations (See \cite{moore_additional_2019} and \cite{usda-aphis_crb_2019}). 

We plan to import CRB from American Samoa and have already provide our collaborator, Dr. Mark Schmaedick at American Samoa Community College, with secure shipping containers we designed to facilitate secure transport of live CRB \cite{moore_container_2017-1}.

\subsection{Laboratory Improvements}

Our current lack of molecular technology for detecting or quantifying OrNV is a major impediment to performing and interpreting bioassays. We have ordered laboratory equipment and supplies to remove this impediment. 


\section{CRB Damage Survey}

The objective of this component of the project is to develop an automated system to evaluate CRB damage by image analysis of roadside video surveys.  We have completed a \textit{proof of concept} trial in which a used deep learning to train an object detector which locates and counts dead and CRB-damaged coconut palms in video streams.  Visual results are presented in a YouTube post \cite{moore_training_2019} and technical details are available in an Open Science Framework Project \cite{moore_open_2019}.

\section{Presentations and Outreach}

\begin{itemize}
	\item During Fall Term 2018, we did monthly presentations on coconut rhinoceros beetles and other insects for the Guam Head Start Program.
	\item An oral presentation was made at the joint annual meetings of the Entomological Society of America and the Canadian Entomological Society, Vancouver: \\\\ \fullcite{moore_failed_2018-1}
	\item An oral presentation and guided field work session was put together for the University of Guam Extension Internship Program: \\\\ \fullcite{moore_coconut_2019-2}	
\end{itemize}

\newpage
\section{Regional Collaboration}

An informal collaboration has been formed among Pacific-based entomologists working on the CRB-G problem. Scientists from Guam, Hawaii, Palau, Papua New Guinea, Solomon Islands, Fiji, Malaysia, Japan and New Zealand have met several times and future meetings are planned:
\begin{itemize}
	\item 2015 Pacific Entomology Conference, Honolulu, HI, USA
	\item 2016 International Congress of Entomology, Orlando, USA
	\item 2017 Japanese Society for Insect Pathology, Tokyo, Japan
	\item 2018 Society for Invertebrate Pathology, Gold Coast, Australia
	\item 2019 XIX International Plant Protection Congress (IPPC2019), India
	\item 2020 (tentative): Pacific Plant Protection Organization, Guam	
\end{itemize}

Although individual institutions involved in finding a solution to the CRB-G problem
have secured funding from multiple, short-term grants, we have been unsuccessful in
finding funding to support a well-coordinated regional project. To facilitate informal
communication and cooperation among members of the current \textit{ad hoc} group, we have
spent some time and resources on developing the following resources.

\subsection{Wiki Site}

During 2018, we built a private wiki intended to facilitate sharing technical information among those working to control CRB-G \cite{moore_crb-g_2019}. It is hoped that this site will be built and maintained by a community of users (ala Wikipedia).

\subsection{Facebook Site}

As an alternative to the wiki, we are launching a controlled-access Facebook site \cite{moore_facebook_2019}.

\subsection{CRB Bibliography}

This bibliography contains over 330 scientific journal articles, technical reports, and presentations about coconut rhinoceros beetle. A copy is provided in Appendix \ref{crb-bibliography}.

\subsection{Online CRB Invasion History Map}

Spread of CRB and CRB-G in the Pacific is being documented using an online, interactive map maintained by Moore \cite{moore_web_2019}. Recent potential geographical expansions include CRB (not the CRB-G biotype) in Vanuata, interception of CRB-G in New Caledonia, and discovery of a single CRB-? adult on Aguiguan Island, a small islanf near Tinian in the Commonwealth of the Northern Mariana Islands.

\newpage
\printbibliography

\newpage
\appendix
\appendixpage
\addappheadtotoc

\section{Recent Bioassay Results}\label{recent-bioassay-results}

\fullcite{moore_bioassay_2019}

\fullcite{moore_bioassay_2019-1}

\fullcite{moore_bioassay_2019-2}

\fullcite{moore_bioassay_2019-3}

\fullcite{moore_bioassay_2019-4}

\fullcite{moore_bioassay_2019-5}

\fullcite{moore_bioassay_2019-6}

\fullcite{moore_bioassay_2019-7}

\fullcite{moore_bioassay_2019-9}

\section{Coconut Rhinoceros Beetle Bibliography}\label{crb-bibliography}

\fullcite{moore_coconut_2019}

\end{document}
