\documentclass[12pt,letterpaper,english,bibliography=totocnumbered]{scrartcl}

\usepackage{indentfirst}
\usepackage{appendix}
%\usepackage{fullpage}
%\usepackage{subfiles}
\usepackage[T1]{fontenc}
\usepackage[latin9]{inputenc}
\usepackage{color}
\usepackage{babel}
\usepackage{verbatim}
\usepackage[unicode=true,pdfusetitle,
 bookmarks=true,bookmarksnumbered=false,bookmarksopen=false,
 breaklinks=true,pdfborder={0 0 0},pdfborderstyle={},backref=false,colorlinks=true]
 {hyperref}
\hypersetup{linkcolor=blue,citecolor=blue,urlcolor=blue}

%\usepackage{pdfpages}
%\usepackage{comment}

\usepackage[backend=biber,maxbibnames=99]{biblatex}
\usepackage{csquotes}
\addbibresource{references.bib}





\begin{document}

\titlehead{USDA APHIS Grant AP18PPQFO000C402\\
Progress Report 2\\
Report ID: AP18PPQFO000-PE-SA2-19\\
Performance Period: March 1, 2019 - August 31, 2019}
\title{Coconut Rhinoceros Beetle Biological Control}
\author{Aubrey Moore, University of Guam}
\maketitle

% The following line provides a link to the source code for this document
Repository: \url{https://github.com/aubreymoore/FB18-Report-2}

Document: \url{https://github.com/aubreymoore/FB18-Report-2/raw/master/report.pdf}

\newpage
\tableofcontents{}




\newpage
\section{Summary}

Coming soon!




\newpage
\section{Background}

The major goal of this project is to find an effective biological
control agent for coconut rhinoceros beetle biotype G (CRB-G). 

Prior to arrival of CRB-G on Guam during 2007, coconut rhinoceros beetle
infestations of Pacific islands were readily controlled by classical
biological control using \textit{Oryctes} nudivirus (OrNV). Following a lack
of response to release of OrNV on Guam, research showed that the Guam
CRB population is a genetically distinct virus-resistant biotype which
has become known as CRB-G\parencite{marshall_new_2017-1}. This biotype is highly invasive and is
causing massive damage to coconut and oil palms in Papua New Guinea
and the Solomon Islands. CRB-G has also invaded Oahu and Rota. Eradication
attempts have been launched on these two islands. 

Additional goals for this project are to establish a CRB damage survey to evaluate efficacy of biocontrol and other tactics, and to maintain and facilitate collaboration with other Pacific island entomologists working to find solutions for CRB-G management. 





\section{Staffing}

Staff for this project currently comprises of only 2 people: the PI, Dr. Aubrey Moore, and a post-doc, Dr. James Grasela. 
\begin{itemize}

    \item Dr. James Grasela, an insect pathologist, has been hired for a term of 2 years with a grant from Department of Interior, Office of Island Affairs.

    \item Ian Iriarte, a graduate student working on this project, resigned to accept a permanent job. Search for a replacement is under way.

\end{itemize} 





\newpage
\section{Bioassays to Detect Candidate Biocontrol Agents for CRB-G}

\cite{grasela_progress_2019}

\subsection{Bioassay Results}

\subsection{CRB Rearing Facility}

\subsection{Laboratory Information System}

\subsection{Acquisition of an OrNV isolate from Taiwan}

\subsection{Acquisition of a Virus-Susceptible CRB Biotype for Comparative Bioassays}

\subsection{Laboratory Improvements}

\begin{comment}

\section{EXTRA}

\subsection{Haemocoel Injection Bioassays}

In this series of assays, we tested 4 islates of OrNV which had produced
by an insect cell culture at the AgResearch Laboratory in New Zealand.
Adult beetles were dosed by direct injection into the haemocoel. This
series is a preliminary test for pathogenicity. Insignificant differences
in mortality curves between virus treatment and the other two treatments
(control treatment and heat-inactivated virus) is an indicator of
pathogenicity. Gut tissue samples have been preserved for histology
and molecular analysis. 

The following is a brief summary of results. Details are provided
in the appended technical reports. Results indicate that isolates
DUG42 and MALB are not pathogenic for CRB-G, but isolates PNG and
V23B are pathogenic. Bioassays in which adult beetles are dosed \emph{per
os} are underway and results will be available in a future report.

\subsubsection{OrNV Isolate DUG42}

Origin: Philippines; 2 replicates; 30 beetles in total

No significant differences among mortality cuves. {[}control, heat-inactivated
virus, virus{]}

\subsubsection{OrNV Isolate MALB}

Origin: Malaysia; 2 replicates; 30 beetles in total

No significant differences among mortality curves. {[}control, heat-inactivated
virus, virus{]}

\subsubsection{OrNV Isolate PNG}

Origin: Papua New Guinea; 4 replicates; 71 beetles in total

Mortality curves in 2 significantly different groups: {[}control,
heat-inactivated virus{]}, {[}virus{]}

\subsubsection{OrNV Isolate V23B}

Origin: Solomon Islands; 4 repicates; 66 beetles in total

Mortality curves in 2 significantly different groups: {[}control,
heat-inactivated virus{]}, {[}virus{]}

\section{Environmental Cabinets and CRB Rearing}

Three environmental cabinets which allow control of temperature, relative
humidity, and lighting for insect rearing were procured and installed.
These chambers are set to maintain 30\textdegree{}C, 80\% RH and
12h photoperiod.

After a power outage caused by a typhoon, one of the cabinets malfunctioned.
It heated beyond the setpoint and killed all beetles. To prevent this
problem from recurring, controllers for all three units have been
programmed to send email to project staff whenever a fault is detected.

The project does not currently rear beetles form egg to adult. Because
CRB are so numerous on Guam, it is far more efficient to field collect
prepupae, pupae and adults and rear these to the age required for
bioassays. Adults are fed banana slices.

\end{comment}


\newpage
\section{CRB Damage Survey}

The objective of this component of the project is to develop an automated system to evaluate CRB damage by image analysis of roadside video surveys.  We have completed a \textit{proof of concept} trial in which a used deep learning to train an object detector which locates and counts dead and CRB-damaged coconut palms in video streams.  Visual results are presented in a YouTube post \cite{moore_training_2019} and technical details are available in an Open Science Framework Project \cite{moore_digital_2019}.

\newpage
\section{Regional Collaboration}

\subsection{Wiki Site}

\subsection{Facebook Site}

\subsection{CRB Bibliography}
\cite{moore_coconut_2019}

\subsection{Online CRB Invasion History Map}


\newpage
\printbibliography

\appendix
%\appendixpage
\addappheadtotoc


\newpage
\section{Coconut Rhinoceros Beetle Bibliography}

\end{document}

